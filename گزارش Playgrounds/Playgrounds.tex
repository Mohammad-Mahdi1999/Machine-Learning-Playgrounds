\documentclass{report}
\usepackage{graphicx}
\usepackage{xepersian}
\settextfont{XB Zar.ttf} 
\usepackage{bidi}
\graphicspath{{/Transformers/Images}} % Change the path
\begin{document}
	\begin{titlepage}
		
		\centering
		\vspace*{2cm}
		\textbf{\LARGE صفحه وب برای دسترسی سریع به Playground ها}
		\vspace{1cm}
		
		
		
		\vfill
		\textbf{استاد: دکتر میرسامان تاجبخش}
		
		
		\textbf{محمد مهدی فرح بخش}
		 
		\vspace{1cm}
		\today
	\end{titlepage}
	
	\begin{center}
		\textbf{
		بسم اللﱣه الرحمن الرحیم
		}
	\end{center}
	
	\section{مقدمه}
در یک صفحه وب به شش شبیه ساز انواع الگوریتم های ماشین لرنینگ دسترسی داشته باشید.
	
	\section{نیازمندی ها و نصب}
	\begin{enumerate}
		\item 
		دستورات زیر را در دایرکتوری
		 \lr{Neual Network Tensorflow playground}
		اجرا کنید:
		\begin{itemize}
			\item \lr{`npm i`} : 
			برای نصب نیازمندی ها 
			\item \lr{`npm run build`} : 
			برای کامپایل
		\end{itemize}
		
		\item 
		برنامه
		\lr{app.py}
		در دایرکتوری 
		\lr{SVM-Visualizer-Web-App}
		بایستی در پسزمینه در حال اجرا باشد و همچنین نیازمندی ها نصب شده باشند.
		\begin{itemize}
			\item \lr{`pip install -r requirement.txt`} : 
			برای نصب نیازمندی ها 
			\item \lr{`py app.py`} : 
			برای اجرا
		\end{itemize}
		
	\end{enumerate}
	

	\section{بخش های برنامه}
	\begin{itemize}
		\item Decision Tree
		
		شما باید بتوانید با استفاده از الگوریتم
		\lr{Decision Tree}
		قارچ های سمی را از خوراکی تفکیک کنید. با انتخاب رئوس مختلف entropy و gini محاسبه شده حاصل از انتخاب رأس را خواهید دید. با هر بار Refresh صفحه با داده های جدید روبرو خواهید شد.

			\begin{figure}
				\centering
				\includegraphics[width=0.5\textwidth]{Decision-Tree.png}
				\caption{نمایی از
					\lr{Decision Tree}
					}
				\label{fig:Decision-Tree}
			\end{figure}
			
		\item k-means
		با کلیک بر روی صفحه داده سمپل ایچاد کنید و از نوار بالا پارامتر های الگوریتم k-means را تنظیم کرده و دکمه Run را بزنید.
				\begin{figure}
					\centering
					\includegraphics[width=0.5\textwidth]{k-means.png}
					\caption{نمایی از
						\lr{k-means}
					}
					\label{fig:k-means}
				\end{figure}
		\item GNN
		در این بخش قدم در دنیای شبکه های عصبی گرافی بگذارید و با برخی مفاهیم آن آشنا شوید.
		\begin{figure}[h]
			\centering
			\includegraphics[width=0.5\textwidth]{GNN.png}
			\caption{نمایی از
				\lr{GNN}
			}
			\label{fig:GNN}
		\end{figure}
		\item KNN
		
		در کنسول زاویه دوربین و داده ها را دستکاری کنید و در محیط سه بعدی اتفاقات را مشاهده نمایید.
		\begin{figure}
			\centering
			\includegraphics[width=0.5\textwidth]{KNN.png}
			\caption{نمایی از
				\lr{KNN}
			}
			\label{fig:KNN}
		\end{figure}
		
		\item SVM
		
		ابتدا نوع مسئله (خطی یا غیر خطی) را انتخاب نموده و با تغییر پارامتر ها اثر هر یک را ببینید. با هر بار رفرش داده ها تغییر می کنند. 
		
			\begin{figure}
			\centering
			\includegraphics[width=0.5\textwidth]{SVM.png}
			\caption{نمایی از
				\lr{SVM}
			}
			\label{fig:SVM}
		\end{figure}
		
		\item \lr{Neural Network TensorFlow playground}
		
		یک شبکه عصبی در اختیار شماست. لایه اضافه کنید هایپر پارامتر ها را تغییر دهید تا به کمترین خطا دست پیدا کنید. 
		
		\begin{figure}
			\centering
			\includegraphics[width=0.5\textwidth]{Neural-Network.png}
			\caption{نمایی از
				\lr{Neural Network}
			}
			\label{fig:Neural-Network}
		\end{figure}
	\end{itemize}
	

	\bibliography{mybib}
	
\end{document}
